\chapter{Einleitung}
Es gibt viele interessante Gadgets mit interessanten Daten für den Alltag: Pulsmesser, Heizungsverbrauch, ... (Sensoren). So auch für Velofahrer. Hier gibt es die Erweiterung des traditionellen Tachometers, der die berechneten Daten auf das Handy ausgibt. 
Die meisten dieser Gadget benötigen eine Batterie. Hier gibt es den neuen Ansatz, die Energie aus der Umgebung zu ernten, auf englisch energy harvesting. In dieser Bachelorarbeit wird ein Tachometer für das Velo entwickelt, der keine Batterie gebraucht. Die Daten dieses Tachometers werden an ein Android-Endgerät gesendet und dort angezeigt.
Fragen: 
Einleitung: Wort Tachometer korrekt ?
Englische Worte: immer kursiv und klein ?

\section{Ausgangslage}
Tachometer auf Handy.
Fahrradcomputer.
- Halterung für das Smartphone
- Speisung Tacho
- Internetrecherche PA \cite{PA_bicycle} 
   "bicycle with a bicyle-mounted energy collector"
   "bicycle electrical generator hub"
   "Electric generator for bicylce"

Fragen:
Welche Aspekte auflisten ?  
Bilder ?

\section{Aufgabenstellung}



\section{Zielsetzung}





\section{Übersicht der Arbeit}
